%%%%%%%%%%%%%%%%%%%%%%%%%%%%%%%%%%%%%%%%%%%%%%%%%%%%%%%%%%%%%%%%%
%                                                               %
%                       Legal Notice                            %
%                                                               %
% This document is copyright (C) Jason Gobat & Darren Atkinson	%
%                                                               %
%%%%%%%%%%%%%%%%%%%%%%%%%%%%%%%%%%%%%%%%%%%%%%%%%%%%%%%%%%%%%%%%%

\begin{appendix}

\newpage{\pagestyle{empty}\cleardoublepage}

\chapter{Installing and Administering \felt{}}
\label{appendix.install}

\section{Building the \felt{} system from source}

The \felt{} package is intended to be easily portable.  It should build on most 
reasonable Un*x systems without any modifications.  To start the build 
{\tt cd} to the toplevel directory of the \felt{} source tree.  From there 
do a {\tt ./configure}.
This will try to automatically determine where to find relevant include files
and libraries on your system and create a new {\tt etc/Makefile.conf} file.  
If you feel comfortable with it you can go in and tweak 
the {\tt Makefile.conf} file by hand if something does not go right.  
The Geompack library used for element generation is written 
entirely in Fortran but C versions as produced by f2c are included in the 
source distribution so there should be no need for you to f2c them yourself.  
If you have a native Fortran compiler and know how to use it in combination
with C code then you can go ahead and play around with the build process to
make this work. The resulting code will probably be faster than if you 
compile the f2c'd code with a C compiler, but not by much, so unless you 
really care it's probably not worth messing with.  You don't need to have the 
f2c libraries 
to use the converted code and the header file f2c.h is provided in the 
distribution. After configuring, do a {\tt make clean} followed by a 
{\tt make all}.  To install the package after it has been successfully 
built do a {\tt make install}.	

The entire package has been compiled and tested under SunOS (4.x and 5.x), 
and Linux (the OS under which it was developed).  This version or earlier 
versions
compiled under HP-UX 8.0 and 9.0 SystemV386 (R3.2.2), and various SGI, DEC, 
and IBM workstations (using Irix, OSF/1 and Ultrix, and AIX) with little or
no problem.  
The files {\em felt.exe} and {\em feltvu.exe} (a graphing application) 
are available as pre-compiled executables for the DOS environment.  	

The most recent version of the source code for \felt{} should always be 
available via anonymous
ftp at cs.ucsd.edu in the directory {\tt /pub/felt}.  We also try to make 
pre-compiled binaries available for a wide variety of platforms and operating
systems; the currently available set of such binaries and how to get to
them are listed in table~\ref{binary_table}.  
\begin{table}
 \begin{center}
 \begin{tabular}{|l|l|l|}
 \hline
Machine type	& operating system	& location \\
 \hline\hline
i386/i486	& Linux 1.0.x		& ftp://cs.ucsd.edu/pub/felt \\
Sun Sparc	& SunOS 4.1.3		& ftp://cs.ucsd.edu/pub/felt \\
Sun Sparc	& SunOS 5.3 (Solaris 2.3) & ftp://cs.ucsd.edu/pub/felt \\
HP 700 series   & HP-UX 9.05		& \\
SGI Indigo	& Irix 5.x		& \\
IBM RS6000	& AIX x.xx		& \\
DEC 5000	& Ultrix x.xx		& \\
DEC Alpha	& OSF/1 x.xx		& \\
i386/i486	& DOS			& ftp://cs.ucsd.edu/pub/felt \\
 \hline
 \end{tabular}
\end{center}
\caption{Type and location of pre-compiled binaries of the \felt{} system.}
\label{binary_table}
\end{table}

If you want to be
made aware of each revision of the \felt{} system then we strongly encourage
you to subscribe to the \felt{} mailing list.  To subscribe send a one line
email message with ``subscribe felt-l'' (without the quotes) in the body
of the message to listserv@mecheng.asme.org.
Only major revisions and changes will be announced broadly (i.e., via Usenet).

\section{Translation files}

The translation files which map non-English terms to the English terms
which \felt{} expects should be installed in the library directory which
you specified during the build.  This path will automatically be
searched when {\em cpp} goes looking for include files (which is really
what the translation files are).  If they are not in some standard place,
users need to specify a {\tt -I} flag on the {\em felt} command line to tell
{\em cpp} where to look for them.  The current filenames are in English,
you'll obviously want to link them or move them to something reasonable
on your system and make users aware of what the appropriate file to include
is called.

The following tables list the currently available translations (that we
have at least).  You can look at the current {\tt german.trn} and 
{\tt spanish.trn} files if you want to change something or provide a new
one.  If you do write a new one, please send it to us.  This is only
a first cut at internationalization, but as we get more translations
we will be in a better position to see how we can make it all work a little
more fluently.  Also, remember that because this is a first cut all you
can really do is type in a \felt{} file by hand using these translations.
Anything {\em velvet} saves or even the file that {\em felt} writes with
the {\tt -debug} command will be in English even if the input file was
given in another language.  All output is still in English.  Lastly, if
you use these translations, you have to use them exactly as they are given
here.  They are not case insensitive, and if there are underscores in the
place of spaces, you must use the underscores.  Things that are user 
specifiable to begin with (object names, problem title) can still be 
whatever you want of course.  If a keyword is not listed in a table below,
use the original English keyword (the English and non-English were probably
equivalent).

{\small
\begin{center}
\begin{tabular}{p{3in}p{3in}} \hline
\bf For German, use \dots	& \bf Instead of \dots \\
\hline 
problembeschreibung 	& problem description \\
titel 			& title \\
knoten 			& nodes \\
elemente 		& elements \\
krafte 			& force \\
kraefte			& forces \\
zwangsbedingung		& constraint \\
zwangsbedingungen 	& constraints \\
materialeigenschaften 	& material properties \\
richtung 		& direction \\
senkrecht 		& perpendicular \\
GlobalesX 		& GlobalX \\
GlobalesY 		& GlobalY \\
GlobalesZ 		& GlobalZ \\
LokalesX 		& LocalX \\
LokalesY 		& LocalY \\
LokalesZ 		& LocalZ \\
werte 			& values \\
verteilte 		& distributed  \\
lasten 			& loads \\
last 			& load \\
gelenk 			& hinge \\
ende 			& end \\
\end{tabular}
\end{center}}

{\small
\begin{center}
\begin{tabular}{p{3in}p{3in}} \hline
\bf For Spanish, use \dots	& \bf Instead of \dots \\
\hline 
descripcion\_del\_problema 	& problem title \\
titulo 				& title \\
nodos 				& nodes \\
elementos 			& elements \\
fuerza 				& force \\
fuerzas				& forces \\
restriccion 			& constraint \\
restricciones 			& constraints \\
propriedades\_del\_material     & material properties \\
direccion 			& direction \\
paralela 			& parallel \\
valores 			& values \\
cargas\_distribuidas 		& distributed loads \\
carga 				& load \\
articulacion\_plana 		& hinge \\
final 				& end \\
\end{tabular}
\end{center}}

\section{Defaults and material databases}

Though the defaults files and material databases are not a necessity for 
{\em velvet}, they are convenient and it might be desirable to install them in 
a location where all users have easy access to them, probably the same
library directory as the translation files.  These files too are really
nothing more than include files which can be included as a convenience for
the user.

%%%%%%%%%%%%%%%%%%%%%%%%%%%%%%%%%%%%%%%%%%%%%%%%%%%%%%%%%%%%%%%%%%%%%%%%%%%

\newpage{\pagestyle{empty}\cleardoublepage}

\chapter{Geompack Error Codes}
\label{appendix.geompack}

\section{Basic description}

Error codes from 1 to 99 indicate not enough space in some array.
Error codes $>=$ 100 mean that the input specification is incorrect,
there is a program bug, or the tolerance TOL is too small or large
(try a different TOLIN value).  

{\small
\begin{tabular}[h]{cp{5.5in}}
\bf Error Code &	\bf Description of error \\ 
 \hline
   1  &  not enough space in EDGE array for routine EDGHT \\
   2  &  not enough space in HOLV array for routine DSMCPR or DSPGDC \\
   3  &  not enough space in 2-D VCL array \\
   4  &  not enough space in HVL array \\
   5  &  not enough space in PVL, IANG arrays \\
   6  &  not enough space in IWK array \\
   7  &  not enough space in WK array \\
   8  &  not enough space in STACK array for routine SWAPEC, DTRIS2, DTRIW2 \\
   9  &  not enough space in TIL array \\
  10  &  not enough space in CWALK array for routine INTTRI \\
  11  &  not enough space in FC array for 3-D \\
  12  &  not enough space in BF array for 3-D \\
  13  &  not enough space in SVCL, SFVL arrays for routine SHRNK3 \\
  14  &  not enough space in 3-D VCL array \\
  15  &  not enough space in FVL, EANG arrays for polyhedral decomposition \\
  16  &  not enough space in FACEP, NRML arrays \\
  17  &  not enough space in PFL array \\
  18  &  not enough space in HFL array \\
  19  &  not enough space in BTL array \\
  20  &  not enough space in VM array \\
  21  &  not enough space in HT array \\
  22  &  not enough space in FC array for K-D \\
  23  &  not enough space in BF array for K-D \\
 100  &  higher primes need to be added to routine PRIME \\
 200  &  abnormal return in routine DIAM2 \\
 201  &  abnormal return in routine WIDTH2 \\
 202  &  parallel lines in routine SHRNK2 \\
 205  &  horizontal ray to right does not intersect polygon in routine ROTIPG \\
 206  &  out-of-range index when popping points from stack in routine VPRGHT \\
 207  &  out-of-range index when scanning vertices in routine VPSCNA \\
 208  &  out-of-range index when scanning vertices in routine VPSCNB \\
 209  &  out-of-range index when scanning vertices in routine VPSCNC \\
 210  &  out-of-range index when scanning vertices in routine VPSCND \\
 212  &  singular matrix in routine VORNBR \\
 215  &  unmatched separator interface edge determined by routine DSPGDC \\
 216  &  all edges of an outer boundary curve are specified as hole 
	 edges in input to routine DSPGDC \\
 218  &  cannot find subregion above top vertex of hole in routine JNHOLE \\
 219  &  angle at hole vertex in modified region is too far from PI due 
	 to use of relative tolerance in routine JNHOLE \\
 222  &  separator not found in routine MFDEC2 (not likely to occur if 
	 ANGSPC $<=$ 30 degrees and ANGTOL $<=$ 20 degrees) \\
 224  &  2 vertices with identical coordinates (in floating point arithmetic)
	 detected in routine DTRIS2 or DTRIW2 \\
 225  &  all vertices input to DTRIS2 or DTRIW2 are collinear \\
 226  &  cycle detected in walk by routine WALKT2 \\
 230  &  invalid (CW-oriented) triangle created in routine TMERGE \\
 231  &  unsuccessful search in routine FNDTRI 
\end{tabular}}

%%%%%%%%%%%%%%%%%%%%%%%%%%%%%%%%%%%%%%%%%%%%%%%%%%%%%%%%%%%%%%%%%
% The GNU Public License                                        %
%%%%%%%%%%%%%%%%%%%%%%%%%%%%%%%%%%%%%%%%%%%%%%%%%%%%%%%%%%%%%%%%%

\newpage{\pagestyle{empty}\cleardoublepage}

\chapter{The GNU General Public License}
\label{appendix.gnu-license}

% Typeset by Matt Welsh (mdw@tc.cornell.edu) for Linux
% Hacked for our purposes because we use the same license 
% This file is covered by its own copyright...

% \chapter{The GNU General Public License}\label{gnu-license}

Printed below is the GNU General Public License (the {\em GPL} or 
{\em copyleft}), under which \felt{} is licensed. It is reproduced here to
clear up some of the confusion about \felt{}'s copyright status---\felt{} 
is {\em not\/} shareware, and it is {\em not\/} in the public domain. The
\felt{} system is copyright \copyright 1993--1995 by Jason Gobat and Darren Atkinson.
Thus, \felt{} {\em is\/} copyrighted, however, you may redistribute it under
the terms of the GPL printed below.

\bigskip
\centerline{{\bf GNU GENERAL PUBLIC LICENSE}}
\centerline{Version 2, June 1991}

Copyright (C) 1989, 1991 Free Software Foundation, Inc.
675 Mass Ave, Cambridge, MA 02139, USA
Everyone is permitted to copy and distribute verbatim copies
of this license document, but changing it is not allowed.

\section{Preamble}
%\centerline{{\sc Preamble}}

The licenses for most software are designed to take away your
freedom to share and change it.  By contrast, the GNU General Public
License is intended to guarantee your freedom to share and change free
software--to make sure the software is free for all its users.  This
General Public License applies to most of the Free Software
Foundation's software and to any other program whose authors commit to
using it.  (Some other Free Software Foundation software is covered by
the GNU Library General Public License instead.)  You can apply it to
your programs, too.

When we speak of free software, we are referring to freedom, not
price.  Our General Public Licenses are designed to make sure that you
have the freedom to distribute copies of free software (and charge for
this service if you wish), that you receive source code or can get it
if you want it, that you can change the software or use pieces of it
in new free programs; and that you know you can do these things.

To protect your rights, we need to make restrictions that forbid
anyone to deny you these rights or to ask you to surrender the rights.
These restrictions translate to certain responsibilities for you if you
distribute copies of the software, or if you modify it.

For example, if you distribute copies of such a program, whether
gratis or for a fee, you must give the recipients all the rights that
you have.  You must make sure that they, too, receive or can get the
source code.  And you must show them these terms so they know their
rights.

We protect your rights with two steps: (1) copyright the software, and
(2) offer you this license which gives you legal permission to copy,
distribute and/or modify the software.

Also, for each author's protection and ours, we want to make certain
that everyone understands that there is no warranty for this free
software.  If the software is modified by someone else and passed on, we
want its recipients to know that what they have is not the original, so
that any problems introduced by others will not reflect on the original
authors' reputations.

Finally, any free program is threatened constantly by software
patents.  We wish to avoid the danger that redistributors of a free
program will individually obtain patent licenses, in effect making the
program proprietary.  To prevent this, we have made it clear that any
patent must be licensed for everyone's free use or not licensed at all.

The precise terms and conditions for copying, distribution and
modification follow.
                    
%\centerline{{\sc GNU GENERAL PUBLIC LICENSE}}
%\centerline{{\sc TERMS AND CONDITIONS FOR COPYING, DISTRIBUTION AND 
%MODIFICATION}}
\section[Terms and Conditions]
{Terms and Conditions for Copying, Distribution, and Modification}

\begin{enumerate}
\item[0.] This License applies to any program or other work which contains
a notice placed by the copyright holder saying it may be distributed
under the terms of this General Public License.  The ``Program'', below,
refers to any such program or work, and a ``work based on the Program''
means either the Program or any derivative work under copyright law:
that is to say, a work containing the Program or a portion of it,
either verbatim or with modifications and/or translated into another
language.  (Hereinafter, translation is included without limitation in
the term ``modification''.)  Each licensee is addressed as ``you''.

Activities other than copying, distribution and modification are not
covered by this License; they are outside its scope.  The act of
running the Program is not restricted, and the output from the Program
is covered only if its contents constitute a work based on the
Program (independent of having been made by running the Program).
Whether that is true depends on what the Program does.

\item[1.] You may copy and distribute verbatim copies of the Program's
source code as you receive it, in any medium, provided that you
conspicuously and appropriately publish on each copy an appropriate
copyright notice and disclaimer of warranty; keep intact all the
notices that refer to this License and to the absence of any warranty;
and give any other recipients of the Program a copy of this License
along with the Program.

You may charge a fee for the physical act of transferring a copy, and
you may at your option offer warranty protection in exchange for a fee.

\item[2.] You may modify your copy or copies of the Program or any portion
of it, thus forming a work based on the Program, and copy and
distribute such modifications or work under the terms of Section 1
above, provided that you also meet all of these conditions:

\begin{enumerate}
\item[a.] You must cause the modified files to carry prominent notices
    stating that you changed the files and the date of any change.

\item[b.] You must cause any work that you distribute or publish, that in
    whole or in part contains or is derived from the Program or any
    part thereof, to be licensed as a whole at no charge to all third
    parties under the terms of this License.

\item[c.] If the modified program normally reads commands interactively
    when run, you must cause it, when started running for such
    interactive use in the most ordinary way, to print or display an
    announcement including an appropriate copyright notice and a
    notice that there is no warranty (or else, saying that you provide
    a warranty) and that users may redistribute the program under
    these conditions, and telling the user how to view a copy of this
    License.  (Exception: if the Program itself is interactive but
    does not normally print such an announcement, your work based on
    the Program is not required to print an announcement.)
\end{enumerate}

These requirements apply to the modified work as a whole.  If
identifiable sections of that work are not derived from the Program,
and can be reasonably considered independent and separate works in
themselves, then this License, and its terms, do not apply to those
sections when you distribute them as separate works.  But when you
distribute the same sections as part of a whole which is a work based
on the Program, the distribution of the whole must be on the terms of
this License, whose permissions for other licensees extend to the
entire whole, and thus to each and every part regardless of who wrote it.

Thus, it is not the intent of this section to claim rights or contest
your rights to work written entirely by you; rather, the intent is to
exercise the right to control the distribution of derivative or
collective works based on the Program.

In addition, mere aggregation of another work not based on the Program
with the Program (or with a work based on the Program) on a volume of
a storage or distribution medium does not bring the other work under
the scope of this License.

\item[3.] You may copy and distribute the Program (or a work based on it,
under Section 2) in object code or executable form under the terms of
Sections 1 and 2 above provided that you also do one of the following:

\begin{enumerate}
\item[a.] Accompany it with the complete corresponding machine-readable
    source code, which must be distributed under the terms of Sections
    1 and 2 above on a medium customarily used for software interchange; or,

\item[b.] Accompany it with a written offer, valid for at least three
    years, to give any third party, for a charge no more than your
    cost of physically performing source distribution, a complete
    machine-readable copy of the corresponding source code, to be
    distributed under the terms of Sections 1 and 2 above on a medium
    customarily used for software interchange; or,

\item[c.] Accompany it with the information you received as to the offer
    to distribute corresponding source code.  (This alternative is
    allowed only for noncommercial distribution and only if you
    received the program in object code or executable form with such
    an offer, in accord with Subsection b above.)
\end{enumerate}

The source code for a work means the preferred form of the work for
making modifications to it.  For an executable work, complete source
code means all the source code for all modules it contains, plus any
associated interface definition files, plus the scripts used to
control compilation and installation of the executable.  However, as a
special exception, the source code distributed need not include
anything that is normally distributed (in either source or binary
form) with the major components (compiler, kernel, and so on) of the
operating system on which the executable runs, unless that component
itself accompanies the executable.

If distribution of executable or object code is made by offering
access to copy from a designated place, then offering equivalent
access to copy the source code from the same place counts as
distribution of the source code, even though third parties are not
compelled to copy the source along with the object code.
  
\item[4.] You may not copy, modify, sublicense, or distribute the Program
except as expressly provided under this License.  Any attempt
otherwise to copy, modify, sublicense or distribute the Program is
void, and will automatically terminate your rights under this License.
However, parties who have received copies, or rights, from you under
this License will not have their licenses terminated so long as such
parties remain in full compliance.

\item[5.] You are not required to accept this License, since you have not
signed it.  However, nothing else grants you permission to modify or
distribute the Program or its derivative works.  These actions are
prohibited by law if you do not accept this License.  Therefore, by
modifying or distributing the Program (or any work based on the
Program), you indicate your acceptance of this License to do so, and
all its terms and conditions for copying, distributing or modifying
the Program or works based on it.

\item[6.] Each time you redistribute the Program (or any work based on the
Program), the recipient automatically receives a license from the
original licensor to copy, distribute or modify the Program subject to
these terms and conditions.  You may not impose any further
restrictions on the recipients' exercise of the rights granted herein.
You are not responsible for enforcing compliance by third parties to
this License.

\item[7.] If, as a consequence of a court judgment or allegation of patent
infringement or for any other reason (not limited to patent issues),
conditions are imposed on you (whether by court order, agreement or
otherwise) that contradict the conditions of this License, they do not
excuse you from the conditions of this License.  If you cannot
distribute so as to satisfy simultaneously your obligations under this
License and any other pertinent obligations, then as a consequence you
may not distribute the Program at all.  For example, if a patent
license would not permit royalty-free redistribution of the Program by
all those who receive copies directly or indirectly through you, then
the only way you could satisfy both it and this License would be to
refrain entirely from distribution of the Program.

If any portion of this section is held invalid or unenforceable under
any particular circumstance, the balance of the section is intended to
apply and the section as a whole is intended to apply in other
circumstances.

It is not the purpose of this section to induce you to infringe any
patents or other property right claims or to contest validity of any
such claims; this section has the sole purpose of protecting the
integrity of the free software distribution system, which is
implemented by public license practices.  Many people have made
generous contributions to the wide range of software distributed
through that system in reliance on consistent application of that
system; it is up to the author/donor to decide if he or she is willing
to distribute software through any other system and a licensee cannot
impose that choice.

This section is intended to make thoroughly clear what is believed to
be a consequence of the rest of this License.

\item[8.] If the distribution and/or use of the Program is restricted in
certain countries either by patents or by copyrighted interfaces, the
original copyright holder who places the Program under this License
may add an explicit geographical distribution limitation excluding
those countries, so that distribution is permitted only in or among
countries not thus excluded.  In such case, this License incorporates
the limitation as if written in the body of this License.

\item[9.] The Free Software Foundation may publish revised and/or new versions
of the General Public License from time to time.  Such new versions will
be similar in spirit to the present version, but may differ in detail to
address new problems or concerns.

Each version is given a distinguishing version number.  If the Program
specifies a version number of this License which applies to it and ``any
later version'', you have the option of following the terms and conditions
either of that version or of any later version published by the Free
Software Foundation.  If the Program does not specify a version number of
this License, you may choose any version ever published by the Free Software
Foundation.

\item[10.] If you wish to incorporate parts of the Program into other free
programs whose distribution conditions are different, write to the author
to ask for permission.  For software which is copyrighted by the Free
Software Foundation, write to the Free Software Foundation; we sometimes
make exceptions for this.  Our decision will be guided by the two goals
of preserving the free status of all derivatives of our free software and
of promoting the sharing and reuse of software generally.

\bigskip
\centerline{{\sc NO WARRANTY}}

\item[11.] BECAUSE THE PROGRAM IS LICENSED FREE OF CHARGE, THERE IS NO WARRANTY
FOR THE PROGRAM, TO THE EXTENT PERMITTED BY APPLICABLE LAW.  EXCEPT WHEN
OTHERWISE STATED IN WRITING THE COPYRIGHT HOLDERS AND/OR OTHER PARTIES
PROVIDE THE PROGRAM ``AS IS'' WITHOUT WARRANTY OF ANY KIND, EITHER EXPRESSED
OR IMPLIED, INCLUDING, BUT NOT LIMITED TO, THE IMPLIED WARRANTIES OF
MERCHANTABILITY AND FITNESS FOR A PARTICULAR PURPOSE.  THE ENTIRE RISK AS
TO THE QUALITY AND PERFORMANCE OF THE PROGRAM IS WITH YOU.  SHOULD THE
PROGRAM PROVE DEFECTIVE, YOU ASSUME THE COST OF ALL NECESSARY SERVICING,
REPAIR OR CORRECTION.

\item[12.] IN NO EVENT UNLESS REQUIRED BY APPLICABLE LAW OR AGREED TO IN WRITING
WILL ANY COPYRIGHT HOLDER, OR ANY OTHER PARTY WHO MAY MODIFY AND/OR
REDISTRIBUTE THE PROGRAM AS PERMITTED ABOVE, BE LIABLE TO YOU FOR DAMAGES,
INCLUDING ANY GENERAL, SPECIAL, INCIDENTAL OR CONSEQUENTIAL DAMAGES ARISING
OUT OF THE USE OR INABILITY TO USE THE PROGRAM (INCLUDING BUT NOT LIMITED
TO LOSS OF DATA OR DATA BEING RENDERED INACCURATE OR LOSSES SUSTAINED BY
YOU OR THIRD PARTIES OR A FAILURE OF THE PROGRAM TO OPERATE WITH ANY OTHER
PROGRAMS), EVEN IF SUCH HOLDER OR OTHER PARTY HAS BEEN ADVISED OF THE
POSSIBILITY OF SUCH DAMAGES.

\end{enumerate}
\centerline{{\sc END OF TERMS AND CONDITIONS}}

%\bigskip
%\centerline{{\sc Appendix: How to Apply These Terms to Your New Programs}}
\section[How to Apply These Terms]
{Appendix: How to Apply These Terms to Your New Programs}

If you develop a new program, and you want it to be of the greatest
possible use to the public, the best way to achieve this is to make it
free software which everyone can redistribute and change under these terms.

To do so, attach the following notices to the program.  It is safest
to attach them to the start of each source file to most effectively
convey the exclusion of warranty; and each file should have at least
the ``copyright'' line and a pointer to where the full notice is found.

\begin{quote}
    \cparam{one line to give the program's name and a brief idea of 
    what it does.}
    Copyright \copyright 19yy  \cparam{name of author}

    This program is free software; you can redistribute it and/or modify
    it under the terms of the GNU General Public License as published by
    the Free Software Foundation; either version 2 of the License, or
    (at your option) any later version.

    This program is distributed in the hope that it will be useful,
    but WITHOUT ANY WARRANTY; without even the implied warranty of
    MERCHANTABILITY or FITNESS FOR A PARTICULAR PURPOSE.  See the
    GNU General Public License for more details.

    You should have received a copy of the GNU General Public License
    along with this program; if not, write to the Free Software
    Foundation, Inc., 675 Mass Ave, Cambridge, MA 02139, USA.
\end{quote}

Also add information on how to contact you by electronic and paper mail.

If the program is interactive, make it output a short notice like this
when it starts in an interactive mode:

\begin{tscreen}
    Gnomovision version 69, Copyright (C) 19yy name of author
    Gnomovision comes with ABSOLUTELY NO WARRANTY; for details type `show w'.
    This is free software, and you are welcome to redistribute it
    under certain conditions; type `show c' for details.
\end{tscreen}

The hypothetical commands `show w' and `show c' should show the appropriate
parts of the General Public License.  Of course, the commands you use may
be called something other than `show w' and `show c'; they could even be
mouse-clicks or menu items--whatever suits your program.

You should also get your employer (if you work as a programmer) or your
school, if any, to sign a ``copyright disclaimer'' for the program, if
necessary.  Here is a sample; alter the names:

\begin{quote}
  Yoyodyne, Inc., hereby disclaims all copyright interest in the program
  `Gnomovision' (which makes passes at compilers) written by James Hacker.

  \cparam{signature of Ty Coon}, 1 April 1989
  Ty Coon, President of Vice
\end{quote}

This General Public License does not permit incorporating your program into
proprietary programs.  If your program is a subroutine library, you may
consider it more useful to permit linking proprietary applications with the
library.  If this is what you want to do, use the GNU Library General
Public License instead of this License.




\end{appendix}
