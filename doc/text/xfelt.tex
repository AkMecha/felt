%%%%%%%%%%%%%%%%%%%%%%%%%%%%%%%%%%%%%%%%%%%%%%%%%%%%%%%%%%%%%%%%%
%                                                               %
%                       Legal Notice                            %
%                                                               %
% This document is copyright (C) Jason Gobat & Darren Atkinson	%
%                                                               %
%%%%%%%%%%%%%%%%%%%%%%%%%%%%%%%%%%%%%%%%%%%%%%%%%%%%%%%%%%%%%%%%%

\newpage{\pagestyle{empty}\cleardoublepage}

\chapter{The {\em xfelt} Application}
\label{xfelt}

\section{Using {\em xfelt}}
\label{xfelt.using}

{\em xfelt} is a simple X-based text editor much like {\em xedit}
(they use the same 
underlying widget after all).  It offers you a way to edit a \felt{} file 
and, without leaving the application, feed that file into the {\em felt} 
application, capture the output from the felt application and display that 
output along with a two- or three-dimensional structural visualization in 
separate windows.  Both output windows can either be saved or printed simply 
by clicking on the appropriate button.	

There are three main menus along the menu-bar.  The first is the {\bf file} 
menu which contains facilities for {\bf open}, {\bf save} and {\bf save as}, 
and {\bf print}, much like any other file menu in any other text editing 
application.  The second is the {\bf edit} menu which also contains some very 
standard, basic editing tools: {\bf cut}, {\bf copy}, {\bf paste}.  The third 
menu, {\bf setup}, allows for configuration of {\em xfelt} and the way 
{\em xfelt} tells {\em felt} to solve the problem.	

Under the {\bf configure} option of the {\bf setup} menu, you can change
the commands that should be used to execute {\em felt} (this should not be
necessary if {\em felt} is located somewhere in your default search path), 
print a text file (e.g. {\tt lp}), print a raw graphics file 
(e.g. {\tt lp -oraw}), 
and convert an xwd file to a printable file (most sites would use {\em xpr} for 
this, consult your system administrator as to which {\em xpr} options you 
should use).  The {\bf felt} options selection under the {\bf configure}
menu allows 
you to toggle {\em felt} command-line flags on and off ({\tt -summary}, 
{\tt -debug}) and specify whether the structure should be visualized in two or 
three dimensions.  		

\section{Solving a problem}

Problems are solved within {\em xfelt} simply by clicking on the {\bf Execute} 
button on the main menu bar.  The entire program will pause until the {\em felt} 
application is done solving the problem (which may be a few minutes depending 
on the size of your problem and the speed of your hardware).  When the 
{\em felt} application is done, the output of {\em felt} will pop-up in a 
separate output window (see section~\ref{felt_prog.output} for information
on how to interpret these results).  

If either of the draw toggle buttons were on, a 
drawing output window will also pop-up with the structure drawn in either 2-d 
or 3-d (if both buttons are on, the structure will be drawn in 2-d).  The 
output of either output window can be saved or printed with the buttons on the 
bottom of the window.  The graphics window will be saved as an xwd file (it 
will prompt for a filename) or printed using the {\em xpr} command and graphics 
printing command given in the configuration (under the {\bf setup} menu).  The 
results window will be saved to a text file (it will prompt for a filename) or 
printed using the printer command given in the configuration.  Clicking 
{\bf Okay} on either window will dismiss the window until the next time 
{\em felt} is run.	

\section{Hints and keyboard shortcuts}

The three basic editing functions have keyboard shortcuts: {\bf cut} is 
\key{alt-w}, {\bf copy} is \key{ctrl-w} and {\bf paste} is \key{alt-y}.  
\key{shift-bkspc} will delete the current selection (without copying it to the 
clipboard). \key{alt-r} is equivalent to pressing the {\bf Execute} button.  
\key{alt-2} and \key{alt-3} toggle two- and three-dimensional structural 
visualization respectively (same functionality as the toggle buttons under the
{\bf felt options} selection from the {\bf setup} menu).  Note that sometimes 
the alt key is referred to as the meta key.
