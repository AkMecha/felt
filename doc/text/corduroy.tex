%%%%%%%%%%%%%%%%%%%%%%%%%%%%%%%%%%%%%%%%%%%%%%%%%%%%%%%%%%%%%%%%%
%                                                               %
%                       Legal Notice                            %
%                                                               %
% This document is copyright (C) Jason Gobat & Darren Atkinson	%
%                                                               %
%%%%%%%%%%%%%%%%%%%%%%%%%%%%%%%%%%%%%%%%%%%%%%%%%%%%%%%%%%%%%%%%%

\newpage{\pagestyle{empty}\cleardoublepage}

\chapter{The {\em corduroy} Application}
\label{corduroy}

\section{Introduction}

{\em corduroy} is a command line application for automatically generating 
nodes and elements.  It takes
a text input file and generates \felt{} format nodes and elements for
inclusion into a \felt{} problem.  Like {\em felt} then, it is a command line
interface to some of the same functionality that you can get in
{\em velvet}.  However, unlike {\em felt} and {\em velvet}, which share their
file syntax (the standard \felt{} syntax), {\em corduroy} has its own
special syntax to describe the way elements are generated.

\section{The {\em corduroy} syntax}
\label{corduroy.syntax}
%%%%%%%%%%%%%%%%%%%%%%%%%%%%%%%%%%%%%%%%%%%%%%%%%%%%%%%%%%%%%%%%%%%%%%%%%%%%
\subsection{Specifying basic parameters}

{\em corduroy} allows you to generate elements in five different ways --
along a single line, as a three-dimensional grid of line elements, as a
two-dimensional grid of four-node planar elements, as a three-dimensional
grid of solid elements and as a two-dimensional mesh of triangular elements.  
Each type of generation is described in its own section; there can be multiple 
types and multiple sections of a given type in any given file.  Besides these
descriptive sections there are only a few basic parameters, all of which
are optional.  Order of specifications does not matter in a {\em corduroy}
file except for numbering of generated nodes and elements and the 
{\tt end} statement which must always come at the end of a file (and is not 
optional).  As in regular \felt{} files, comments can be denoted with 
{\tt /*} and {\tt */} as in the C programming language and symbolic expressions
may be substituted for any non-integer value.

You can use {\tt start-node=} and {\tt start-element=} to specify the number
for the first things that get generated.  Normally, {\em corduroy} starts
the numbering at one and then numbers sequentially as it moves from section
to section of your input file.  These two specifications allow you to override
these starting points in situations where you may already have a few elements
and nodes defined and you are going to attach the things that get generated
after {\em corduroy} has done its work.

{\tt constraint=} and {\tt material=} will assign a constraint or material
name to the initial node and element.  This is simply a convenience; you will
need to be sure to go back and actually define these objects after everything
is generated and you are putting the finishing touches on the problem 
definition.

\subsection{Generating elements along a line}

The simplest case of element generation is a line which is divided up into
an arbitrary number of elements.  You might use something like this if you
were examining the accuracy of a cantilever beam model and wanted to use
successively higher numbers of elements from one case to the next.  The
{\em corduroy} specification for a {\tt line} section looks like this
\begin{screen}
 \begin{verbatim}
line
element-type = beam
start        = (0,0,0)    
end          = (5,5,5)    
number       = 20
 \end{verbatim}
\end{screen}
This example would generate twenty beam elements all lying along the line from
the origin to a point at x=5, y=5, z=5 in rectangular Cartesian coordinates.
The default element type for both line and grid generation is truss.  Valid
type names are the same as in a standard \felt{} file (see 
chapter~\ref{problem}).

\subsection{Generating a grid of line elements}

Another relatively simple case is when you want to generate a three-dimensional
grid of line elements. An excellent example of when this might be useful is
for a model of a steel frame structure. The specification for a grid is
\begin{screen}
 \begin{verbatim}
grid
element-type = beam
start        = (0,0,0)   end      = (100,500,100)
x-number     = 4         y-number = 20              z-number = 4
 \end{verbatim}
\end{screen}
This example would generate a three-dimensional frame structure with four
bays along both the x and z axes and 20 bays along y axis.

\subsection{Generating a grid of quadrilateral planar elements}

The case of a simple grid of planar four-node elements is almost identical
to the grid of line elements described above.  The only difference is that
there are no specifications along the z direction for quadrilateral grids.
An example of a mesh for a long rectangular plate using {\tt htk} plate
bending elements (the default element type for quadrilateral grids is
quad\_PlaneStress) might look like:
\begin{screen}
 \begin{verbatim}
quadrilateral grid
element-type = htk
start        = (0,0)      end      = (10,3)
x-number     = 20         y-number = 6
 \end{verbatim}
\end{screen}
The result would be a rectangular region meshed with 120 equi-sized square
elements.

\subsection{Generating a grid of solid brick elements}

The case of a grid of solid eight-node elements is also very similar to
grids of line and planar elements described in the preceding two sections.
An example of a mesh for a long rectangular plate using {\tt brick} 
elements (the default type for grids of solid elements) might look like:
\begin{screen}
 \begin{verbatim}
brick grid
start        = (0,0,0)      end      = (8,12,6)
x-number     = 4            y-number = 6            z-number=3           
 \end{verbatim}
\end{screen}
The result would be a three-dimensional solid region three elements deep,
four elements wide, and six elements tall.

\subsection{Grid spacing rules}
\label{grid_spacing_rules}

For all of the line and grid generation examples given above, we
could also have directed that {\em corduroy} generate the elements
with non-linear spacing rules.  Such spacings are common when we
know that we want a higher mesh density at one corner of the grid
or only at the end of the line.  The rules available are 
{\tt linear} (this is the default, equi-spaced), {\tt sinusoidal},
{\tt cosinusoidal}, {\tt logarithmic}, {\tt parabolic}, 
{\tt reverse-parabolic}, and {\tt reverser-logarithmic}.
If we are placing the sides of $n$ elements along a line, then we
need to locate $n + 1$ nodes on the line.  The coordinate $x_i$ of
each node is defined as follows for each of the spacing rules
\begin{eqnarray}
  {\rm linear:}              & x_i = & L \beta \\
  {\rm cosinusoidal:}        & x_i = & L - L\cos\left({\pi \over 2} \beta \right)  \\
  {\rm sinusoidal:}          & x_i = & L \sin\left( {\pi \over 2}  \beta \right)  \\
  {\rm logarithmic:}         & x_i = & L \log_{10} \left( 1 + 9\beta \right) \\
  {\rm parabolic:}           & x_i = & L \beta^2 \\
  {\rm reverse-logarithmic:} & x_i = & L - L \log_{10} \left( 10 - 9\beta \right) \\
  {\rm reverse-parabolic:}   & x_i = & L\sqrt{ \beta } \\
\end{eqnarray}
where $\beta = (i - 1)/n$.  

\subsection{Generating a triangular mesh}

The triangular mesh generation capabilities of {\em corduroy} use
Jonathan Shewchuk's Triangle routine.  In addition to specifications
that describe the boundary and the holes of your two-dimensional region,
you must also define the approximate number of elements to generate
and a constraint on the maximum area of any one generated element.
The number of elements is specified with {\tt target=}.  The area
constraint is given using {\tt alpha=}, where $\alpha$ is
defined such that 
\begin{equation}
A_{\rm elt} \leq {\alpha A_{\rm total} \over {\rm total}}.
\end{equation}
The entire {\em corduroy} input file to generate the nodes and elements for the
wrench model pictured in Figure~\ref{velvet.main} would look like this.
\begin{screen}
 \begin{verbatim}
start-node    = 1        /* unnecessary as this is the default */
start-element = 1

triangular mesh
element-type = CSTPlaneStress
angtol = 20    dmin = 0.5     min = 100		/* except for min and max */
angspc = 30    kappa = 0.25   max = 300		/* these are defaults     */

boundary = [
   (0,50)
   (20,10)
   (40,0)
   (40,50)
   (55,70)
   (95,70)
   (115,50)
   (115,0)
   (135,10)
   (155,50)
   (155,90)
   (115,160)
   (170,404)
   (122,404)
   (67,160)
   (0,90)
]

end
 \end{verbatim}
\end{screen}

To generate a mesh in a rectangular plate with two side by side rectangular 
holes in it, the input file could look like
\begin{screen}
 \begin{verbatim}
triangular mesh          /* use all defaults, including CSTPlaneStress */
boundary = [
   (0,0)
   (20,0)
   (20,10)
   (0,10)
]

hole = [
   (6,3)
   (8,3)
   (8,7)
   (6,7)
]

hole = [
   (12,3)
   (14,3)
   (14,7)
   (12,7)
]

end
 \end{verbatim}
\end{screen}
Note that the hole definitions do not need to come before the boundary, but
that points must always be specified in a counter-clockwise order.

\section{Using {\em corduroy}}
\label{corduroy.using}

The only options to {\em corduroy} are the standard {\em cpp} options
as in both {\em felt} and {\em velvet} (like those two programs, 
{\em corduroy} runs all of its input through the pre-processor before
actually operating on it) and a {\tt -debug} flag which causes
{\em corduroy} to reproduce as output what it thinks it received for
input.  To generate, all you need to do is create
an input file with a text editor and then run {\em corduroy} with the
name of this file as the last argument on the command line.  Output
will be directed to standard out.  So if you had a generation description
in a file {\tt foo.crd}, you could turn it into the basics of a \felt{}
file called {\tt foo.flt} with the following command
\begin{screen}
 \begin{verbatim}
% corduroy foo.crd > foo.flt
 \end{verbatim}
\end{screen}

\section{Incorporating output into a \felt{} file}
\label{corduroy.incorporating}

Because {\em corduroy} only generates the node and element sections of
a \felt{} file, you still have some editing to do before you can call the
problem completely defined and actually try to solve it with {\em felt}
or {\em velvet}. In general, you will have to define forces and constraints
to assign to nodes and material properties (and possibly distributed
loads) to assign to elements.  Remember, the {\tt material=} and 
{\tt constraint=} specification are only a convenience.  These two names
will simply be assigned to the first element and node, respectively (and
all the following elements and nodes by default).  You still need to 
actually define those objects and assign any different objects to
appropriate nodes and elements.  To do all this defining and assigning you 
can either use your favorite text editor or you can load 
the problem into {\em velvet} and do it all from there.  

One thing to keep in mind whenever you use {\em corduroy} to generate
a problem for you is that {\em corduroy} is not very good at node numbering.
Using the renumbering
features of either {\em felt} or {\em velvet} is highly recommended with
{\em corduroy} generated problems.  In general the problems that you will
use {\em corduroy} for will be quite large and thus the reduction in memory
and solution time would be significant even if node numbering had been
intelligent to start with; the reductions for large problem with initially
very bad numbering can be remarkable.
