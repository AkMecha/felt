%%%%%%%%%%%%%%%%%%%%%%%%%%%%%%%%%%%%%%%%%%%%%%%%%%%%%%%%%%%%%%%%%
%                                                               %
%                       Legal Notice                            %
%                                                               %
% This document is copyright (C) Jason Gobat & Darren Atkinson	%
%                                                               %
%%%%%%%%%%%%%%%%%%%%%%%%%%%%%%%%%%%%%%%%%%%%%%%%%%%%%%%%%%%%%%%%%

\newpage{\pagestyle{empty}\cleardoublepage}

\chapter{The Algorithms Behind \felt{}}
\label{algorithm}

\section{Some background}
In general, there is a massive base of experience in the numerical analysis 
community that relates to programming the finite element method.
Any time you see the method implemented, chances are that the implementation
is in some piece of software. So why is it that good books on computational
techniques are hard to come by and good books on the underlying theory
and mathematics abound?  Go figure.  
As we mentioned in chapter~\ref{analysis},
if you need some references to the latter we like Hughes' \cite{hughes:fem}
and Zienkiewicz and Taylor's \cite{zienk:taylor:fem} ``classic'' textbooks.
Logan's \cite{logan:fem} book is a good introductory text.  He
approaches the method from the same direction that many of us come to finite
elements, through classical matrix structural analysis.  He's a bit lean
on the mathematical basis and theory, but that's probably why it makes a
good introductory text.  It is probably our ignorance, but we are not aware of 
any definitive texts on the algorithmic implementation of the finite element 
method.  Texts that have been recommended, but that we have not really
worked with include Segerlind \cite{segerlind:fea}, a very 
well-recommended introductory text by Burnett \cite{burnett:fea} and a book by
Hinton and Owen \cite{hinton:fep}.  

\section{Elementary C programming}
\label{algorithm.c}

The advantages to coding \felt{} entirely in C rather than the more traditional 
(at least for finite element analysis) choice of Fortran are: 1.~keeping the 
package consistent -- the system and graphical interface stuff required working 
in C, having all of the mathematics in C made interfacing the two aspects a 
lot easier. 2.~C's flexibility in working with data structures makes it really 
easy to implement some of the concepts of finite element analysis in the 
code -- elements have pointers to materials and loads, nodes have pointers to 
forces and constraints.  There is no need to keep track of lots of separate 
arrays of things; generally speaking, passing the node and element arrays 
around is all the information that a routine will need (or even just passing 
the elements around as they contain pointers into the nodes array).  	

For those unfamiliar with C, here's a quick and dirty lesson that might help 
you read the code with a little more comprehension.  Basically, in C you can 
declare new types of variables, including entire data structures.  What this 
means is that we can define a structure called an Element which has several 
different members (think of each member as a different field or record in a 
database).  When we declare a variable to be of type Element, that variable 
then contains all these different fields and we can get at all the information 
for that element through that one variable.  For instance, in \felt{} an 
element structure contains all of the following:

{\small
\begin{screen}
 \begin{verbatim}
Element {	
    unsigned    number;          /* this element's global number           */
    Node        node [ ];        /* array of pointers to element's nodes   */
    Matrix      K;               /* element stiffness matrix               */
    Definition  definition;      /* definition structure (type) of element */ 
    Material    material;        /* pointer to material property           */	
    Distributed	distributed [4]; /* array of distributed loads             */ 	
    unsigned    numdistributed;  /* number of distributed loads assigned   */
    Stress      stress [ ];      /* element stresses                       */
    unsigned    ninteg;          /* number of integration points           */
}
 \end{verbatim}
\end{screen}}

Several of these members are structures themselves: {\tt Node}, {\tt Matrix}, 
{\tt Distributed}, {\tt Definition}, {\tt Material} and {\tt Stress} are 
other structures used in \felt{}.	

Because most of the \felt{} routines access the structures as pointers (that is 
usually when the code references a structure it is actually referencing the 
address of that structure in memory ... this is something you shouldn't need 
to worry about to simply read and understand the mathematics), the usual way 
that you will see code that refers to elements is like this: 
\mbox{{\tt element -> node[1] -> x}}, or \mbox{{\tt element -> material -> E}}. 
The first of these 
examples gets the value of the x-coordinate of the element's first node 
(not necessarily globally numbered node 1).  The second gets the value of the 
Young's modulus of the material assigned to this element.  If the array of 
elements is under consideration, you might seem something like this: 
\mbox{{\tt element[3] -> stress[1] -> values[1] }}.  This will be the value of 
the first stress component at the first integration point on element 3.  The 
other common data structures in \felt{} are:

{\small
\begin{screen}
 \begin{verbatim}
/* A reaction force */

typedef struct reaction {
    double   force;                     /* reaction force             */
    unsigned node;                      /* node number                */
    unsigned dof;                       /* affected degree of freedom */
} *Reaction;

/* Element stress */

typedef struct stress {
    double  x;                          /* x coordinate           */
    double  y;                          /* y coordinate           */
    double  z;                          /* z coordinate           */
    double  values [ ];                 /* computed stress values */
} *Stress;

/* An element definition */

typedef struct definition {
    char      *name;             /* element symbolic name              */
    int      (*setup) ( );       /* element initialization function    */ 
    int      (*stress) ( );      /* stress computation function        */
    Shape     shape;             /* element dimensional shape          */
    unsigned  numnodes;          /* number of nodes in element         */
    unsigned  shapenodes;        /* number of nodes which define shape */
    unsigned  numstresses;       /* number of computed stress values   */
    unsigned  numdofs;           /* number of degrees of freedom       */
    unsigned  dofs [7];          /* degrees of freedom                 */
    unsigned  retainK;           /* retain element K after assemblage  */
} *Definition;

/* A distributed load */

typedef struct distributed {
    char     *name;             /* name of distributed load */
    Direction direction;        /* direction of load        */
    unsigned  nvalues;          /* number of values         */
    Pair      value [ ];        /* nodes and magnitudes     */
} *Distributed;

/* A force */

typedef struct force {
    char     *name;                /* name of force           */
    VarExpr   force [7];           /* force vector            */
    VarExpr   spectrum [7];        /* vector of input spectra */ 
} *Force;

/* A constraint */

typedef struct constraint {
    char   *name;                /* name of constraint            */
    char    constraint [7];      /* constraint vector             */
    double  ix [7];              /* initial displacement vector   */
    double  ax [4];              /* initial acceleration vector   */
    double  vx [4];              /* initial velocity vector       */
    VarExpr dx [7];              /* boundary displacement vector  */
} *Constraint;

/* A material */

typedef struct material {
    char  *name;                /* name of material                       */
    double E;                   /* Young's modulus                        */
    double Ix;                  /* moment of inertia about x-x axis       */
    double Iy;                  /* moment of inertia about y-y axis       */
    double Iz;                  /* moment of inertia about z-z axis       */
    double A;                   /* cross-sectional area                   */
    double J;                   /* polar moment of inertia                */
    double G;                   /* bulk (shear) modulus                   */
    double t;                   /* thickness                              */
    double rho;                 /* density                                */
    double nu;                  /* Poisson's ratio                        */
    double kappa;               /* shear force correction                 */
    double Rk;                  /* Rayleigh stiffness damping coefficient */
    double Rm;                  /* Rayleigh mass damping coefficient      */ 
    double Kx;                  /* conductivity in x direction            */
    double Ky;                  /* conductivity in y direction            */
    double Kz;                  /* conductivity in z direction            */
    double c;                   /* heat capacity                          */
} *Material;

/* A node */

typedef struct node {
    unsigned   number;          /* node number                    */
    Constraint constraint;      /* constrained degrees of freedom */
    Force      force;           /* force acting on node           */
    double     m;               /* lumped mass at this node       */
    double     eq_force [ ];    /* equivalent force               */
    double     dx [7];          /* displacement                   */
    double     x;               /* x coordinate                   */
    double     y;               /* y coordinate                   */
    double     z;               /* z coordinate                   */
} *Node;
 \end{verbatim}
\end{screen}}

Hopefully the organization of the data structures is intuitive enough to 
someone familiar with finite elements that a detailed understanding of the 
mechanics of the C language will not be necessary.  		

\section{Introduction to the general \felt{} routines}

Throughout \felt{}, simplicity and readability have generally won out over 
speed and efficiency (at least where the mathematics are concerned ... the 
system and GUI code are another story entirely).
What this means is that \felt{} does not make use of many of the algorithmic 
tricks that have been developed over the years for working with finite 
elements.  The most glaring example of this is probably the fact that the 
\felt{} matrix manipulation library is fairly simple and does not make use of 
symmetry in most cases.  

\section{Details of a few general \felt{} routines}

\subsection{Finding the active DOF}
The first step in the solution of a \felt{} problem is to 
find all the degrees of freedom that the different types of elements in 
the problem use.  This information gets stored in two different six-element 
arrays.  In the {\tt dofs\_pos} array a non-zero value 
in the {\em i}th position indicates that the {\em i}th DOF is 
active.  The actual number in the {\em i}th position indicates which DOF this 
is for the current problem (i.e. in a problem consisting solely of beams, a 3 
in the sixth position indicates that rotation about the z-axis is the third 
DOF in the current problem).  The second array is {\tt dofs\_num}.  In
this array, the entries from one through the number of active DOF
are all non-zero.  A value of {\em j} in the {\em i}th position indicates
that the standard DOF {\em i} is the {\em j}th DOF for this problem (i.e.,
for our example problem consisting of beams only, a 6 in the third
position indicates that the third active DOF for this problem is rotation
about the z-axis).  Both of these arrays are available globally through the 
{\tt analysis} structure. 

\subsection{Node renumbering}
We all know that the profile and bandwidth of a global stiffness or mass
matrix -- critical factors in determing memory requirements and ultimate
solution speed -- are a function of the global node numbering.  The 
contribution from each element is assembled into the global matrices
based on the global numbers of the elements' nodes.  Better node numbering
schemes take this into account and try to keep the contribution from each
element as near to the diagonal of the global matrix as possible.  Given
the compact column storage scheme described below, a matrix with most
entries very near the diagonal can use significantly less storage than
a very full matrix. The linear equation solver can also make use of this
reduced matrix and operate a lot faster by not having to operate on a lot
of zero entries. 

There are numerous algorithms available to try to optimize node numbering
with just such a goal in mind.  The one that we use in \felt{} is popularly
known as
Gibbs-King \cite{gk:renumbering}, the profile reduction variant of the
Gibbs-Poole-Stockmeyer algorithm \cite{gps:renumbering} which primarily
tries to minimize bandwidth.  Other popular algorithms include 
Cuthill-Mckee \cite{cm:renumbering} and 
Reverse Cuthill-McKee \cite{rcm:renumbering}.

\subsection{Assembling the global stiffness matrix}
This step of the process in FElt actually accomplishes three things.
For each element, this routine calls an element setup routine
based on element type.  Besides filling out \mbox{{\tt element -> K}} most
element setup routines will also calculate the equivalent
nodal forces on the element's nodes if appropriate (i.e., if there is
a distributed load on the element).  

Assembling the stiffness and mass matrices in a transient analysis problem
works the same way.  For transient analysis, the element setup routines
calculate a mass matrix according to the the mass-mode set in the analysis
parameters section.  The global stiffness and mass matrices are formed
exactly as in the static case.  Nodal lumped masses are superposed after
the element mass matrices have been assembled.  

The damping matrix can be formed in either of two ways -- both of them
based on a Rayleigh damping model.   In the first method, the damping
matrix is assembled at the same time as the stiffness and mass matrices
by creating and assembling element damping matrices based on the 
Rayleigh damping coefficient for each element's material, 
\begin{equation}
{C_{e}} = {{R_{k}}^{e} K_{e} + {R_{m}}^{e} M_{e}}.
\end{equation}
Where ${R_k}^e$ and ${R_m}^e$ are the Rayleigh damping coefficients for 
the given material.  Alternatively, if non-zero $R_k$ and $R_m$ values are
given in the analysis parameters of the \felt{} file, then these coefficients
will be applied to form a global damping matrix only after the global
mass and stiffness matrices have been completely assembled,
\begin{equation}
{C} = {R_{k} K + R_{m} M}.
\end{equation}

\subsection{Compact column representation}
All of the global matrices are assembled directly into a compact 
column representation.  This representation tries to minimize storage
requirements by taking each column of the matrix and only storing the
entries from the first non-zero entry to the diagonal.  In this scheme
the following $6\times6$ would require a vector of length 14 to store
as opposed to a matrix with 36 entries,
$$
\def\tempa{\multicolumn{1}{c|}{0}}
\def\tempb{\multicolumn{1}{c|}{1}}
\left[
\begin{array}{cccccc}
\cline{1-2} 
x & \tempb & 0 & 0 & 0 & 0 \\ \cline{3-3} \cline{5-5}
  & x & \tempb & \tempa & \tempb & 0 \\ \cline{4-4}
  &   & x & x & \tempb & 0 \\ \cline{6-6}
  &   &   & x & 0 & 1 \\
  &   &   &   & x & x \\
  &   &   &   &   & x 
\end{array}
\right].
$$

Assembly is done in two passes through
the elements.  The first pass is used to optimize the storage scheme
by finding out just how tall each column needs to be. Given this information,
vectors are actually allocated to hold the global matrices.  This
first pass is also used to construct a vector of diagonal addresses. For each
row of what would be the full matrix, this vector contains the position
in the compact form of the diagonal of that row.  For our above example,
that vector would be 
$$
\left[ 
\begin{array}{cccccc}
1 & 3 & 5 & 7 & 11 & 14 
\end{array}
\right].
$$
The second pass of the assembly process actually inserts the appropriate 
contribution from each element into these vectors.  

Any routine that needs to access the global stiffness or mass matrices
can behave just as if they were full two-dimensional matrix representations.
We achieve this transparency through calls to a function which uses the
information in the diagonal addresses array (which is attached as a member 
of the standard matrix data structure) to convert
from a row-column location to a single address in the global vector
representations.  

\subsection{Dealing with boundary conditions}

\felt{} can handle several types of boundary conditions.  The simplest
is a fixed DOF at a given node.  In this case, the row and column of the
global stiffness matrix corresponding to the fixed DOF are simply filled with 
zeros and a one is placed on the diagonal at that location; a zero is placed
in the appropriate DOF of the global force vector just for clarity.  

The second type of condition is a displacement condition such as might
be found in a settlement of support problem.  In this case, before we 
eliminate the rows and columns of the stiffness matrix, we need to adjust
the force vector to account for the displacement.  Given a globally
numbered DOF, $n$, which has a displacement condition, $dx$, global force 
vector, $F$, and global stiffness matrix, $K$, then for each DOF, $i$, in 
the problem, 
\begin{equation}
F(i) = F(i) - K(i,n)dx.
\end{equation}
After this adjustment, the rows and columns of $K$ associated with DOF $n$
can be zeroed out as in the ordinary fixed case.  The magnitude of the
displacement condition, $dx$, is placed in the $F(n)$ to insure that
$dx$ will re-appear exactly in the final solution for nodal displacements.

The final type of condition is a hinge and because the primary adjustments
that this condition requires were made in the element stiffness matrix 
routines, all we do here is zero out rows and columns of the stiffness matrix
and force vector as in the fixed case.

\subsection{Solving for nodal displacements}
The \felt{} routine to solve for nodal displacements (or to solve
any linear system of equations) takes the
compact column representation and solves $Kd=F$ for the global displacement 
vector using a skyline solver. This procedure basically involves 
a Crout factorization and forward and backward substitution on the compact
column representation of the global stiffness matrix.  	

\subsection{Time integrating in transient structural analysis}
\label{algorithms.analysis}

Time-stepping in a transient structural analysis problem is done with the
Hilbert-Hughes-Taylor alpha variant of Newmark integration.  There are
three critical parameters in this algorithm.  The first two, $\beta$
and $\gamma$, are the standard parameters of Newmark integration.
Depending on the values for these two parameters we can implement
several well-known algorithms.  Setting $\beta = {1 \over 4}$ and 
$\gamma = {1 \over 2}$ results in an unconditionally stable average
acceleration (trapezoidal rule) implementation.  $\beta = {1 \over 6}$ and 
$\gamma = {1 \over 2}$ results in a linear acceleration algorithm.

The HHT-$\alpha$ algorithm adds a third parameter, $\alpha$, to account for the 
decrease in order of accuracy that results when you introduce numerical
damping into the Newmark method.  Setting $\alpha = 0$ reduces the
problem to a standard Newmark method.  Choosing $\alpha \in {[-{1 \over 3}, 0]}$,
$\gamma = (1 - 2\alpha)/2$ and $\beta = (1 - \alpha)^2/4$ results in 
an unconditionally stable, second-order accurate algorithm \cite{hughes:fem}.

With these three parameters, the time-discrete equation of motion in the 
HHT-$\alpha$ method is written as
\begin{equation}
\label{discrete_motion}
Ma_{i+1} + (1 + \alpha) C v_{i+1} - \alpha C v_{i} + 
(1 + \alpha) K d_{i+1} - \alpha K d_{i} = F \left( t_{i+1} + {\alpha}{\Delta}t \right).
\end{equation}
The standard Newmark finite difference formulas are used as approximations
for $d_{i+1}$ and $v_{i+1}$,
\begin{eqnarray}
d_{i+1} & = & {\tilde d}_{i+1} + \beta \Delta t^{2} a_{i+1}, \label{newmark_d} \\
v_{i+1} & = & {\tilde v}_{i+1} + \gamma \Delta t a_{i+1}, \label{newmark_v}
\end{eqnarray}
where the predictor variables, ${\tilde d}_{i+1}$ and ${\tilde v}_{i+1}$
are defined as
\begin{eqnarray}
{\tilde d}_{i+1} & = & d_{i} + \Delta t v_{i} + 
                       \Delta t^{2} \left({1 \over 2} - \beta \right) a_{i}, \\
{\tilde v}_{i+1} & = & v_{i} + \left(1 - \gamma \right) \Delta t a_{i}.
\end{eqnarray}
We can form an implicit update equation with $d_{i+1}$ as the only unknown
by rearranging equation~\ref{newmark_d} and substituting the result along
with equation~\ref{newmark_v} into the equation of motion 
(equation~\ref{discrete_motion}).  If we denote the left hand coefficient
matrix as $K'$ and the right hand contributions not due to the force vector
as $F'$ then the update equation is
\begin{equation}
{K'} d_{i+1} = {F'}_{i+1} + F \left( t_{i+1} + \alpha \Delta t \right).
\end{equation}
where
\begin{equation}
K' = M + \gamma \Delta t C + (1 + \alpha) \beta \Delta t^{2} K,
\end{equation}
and
\begin{equation}
{F'}_{i+1} = \left[ M + \gamma \Delta t C \right] {\tilde d}_{i+1} 
     - (1 + \alpha) \beta \Delta t^{2} C {\tilde v}_{i+1} 
     + \alpha \beta \Delta t^{2} \left[ C v_{i} + K d_{i} \right].
\end{equation}

\subsection{Time integrating in transient thermal analysis} 

As in transient structural analysis, the time-stepping in a transient
thermal analysis problem is done using an approach similar to Newmark integration.
In this case, however, there is only one integration parameter, $\alpha$,
because of the simpler nature of the governing ODE in the thermal
analysis case.  Using a generalized trapezoidal rule we express the 
temperature vector update equation as
\begin{equation}
{T_{i+1}} = {T_{i} + \Delta t \left[{\left(1 - \alpha \right) {\dot T}_{i} + \alpha {\dot T}_{i+1}}\right]}.
\label{temperature_update}
\end{equation}
Through some algebraic manipulations of the governing equation at time
steps $i$ and $i+1$ and subsitution of equation~\ref{temperature_update}
to eliminate time derivative terms, we can write the implicit update 
equation for $T$ as
\begin{equation}
{K' T_{i+1}} = {\left[M - \Delta t \left(1 - \alpha \right) K \right] T_{i} + 
               \left(1 - \alpha \right) \Delta t F_{i} + 
               \alpha \Delta t F_{i+1}},
\end{equation}
where 
\begin{equation}
{K'} = {M + \alpha \Delta t K},
\end{equation}
and everything on the right-hand side is known.

Logan~\cite{logan:fem} gives the following summary of the methods that
result from various choices of $\alpha$:
$\alpha = 0$, simple forward difference scheme which is only conditionally 
stable;  $\alpha = {1 \over 2}$ Crank-Nicolson or trapezoidal rule which is 
unconditionally stable; $\alpha = {2 \over 3}$, Galerkin which is also 
unconditionally stable; $\alpha = 1$, backward difference which is 
unconditionally stable.

\subsection{Solving the eigenvalue problem}

\felt{} uses a relatively unsophisticated procedure to solve for the 
eigenvalues and eigenvectors in a modal analysis problem.  The first step is
to actually remove constrained DOF from the global stiffness and mass
matrices.  This procedure is less efficient than the simple zeroing out of
constrained DOF used in transient and static analysis, but it is necessary
for the actual numerical algorithm used to solve the eigenvalue problem.

In terms of the reduced mass, $M$, and stiffness, $K$, matrices, the 
generalized eigenvalue problem can be written
\begin{equation}
\label{general_eigenvalue}
K x = \lambda M,
\end{equation}
where $\lambda$ and $x$ are the eigenvalue and eigenvector in a given mode.
We can transform this to standard form, 
\begin{equation}
\left( A - \lambda I \right) x = 0,
\end{equation}
by forming the Cholesky factorization of the mass matrix, $M = QQ^{T}$,
and realizing that $Q^{-T} M Q^{-1} = I$.  With this factorization, we just 
need to make the substitution $x = Q^{-1} {\hat x}$ and pre-multiply both sides
of equation~\ref{general_eigenvalue} by $Q^{-T}$; the problem then becomes
\begin{equation}
Q^{-T} K Q^{-1} {\hat x} = \lambda I {\hat x}
\end{equation}
which is now in standard form.  \felt{} solves this standard eigenvalue problem
by applying the QL method to the tridiagonal form of the transformed 
coefficient matrix.

Note that this procedure will not work if there are zeros on the diagonal
of the mass matrix (the global mass matrix must be positive-definite for the
Cholesky factorization to be non-singular).  This kind of condition often 
occurs in problems with lumped mass formulations of the element mass matrices.  
